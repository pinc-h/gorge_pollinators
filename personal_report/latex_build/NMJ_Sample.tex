%% Last edited July 16th 2024
%% A personal report of my work and findings working with pollinator insects in Esquimalt Gorge Park, separate from the internal and public reports I aided in writing for the Gorge Waterway Action Society.

\documentclass{CUP-JNL-NMJ}%


%%%% Packages
\usepackage{graphicx}
\usepackage{multicol,multirow}
\usepackage{amsmath,amssymb,amsfonts}
\usepackage{mathrsfs}
\usepackage{amsthm}
\usepackage{rotating}
\usepackage{appendix}
\usepackage[numbers]{natbib}
\usepackage{ifpdf}
\usepackage{xcolor}
\usepackage[colorlinks,allcolors=blue]{hyperref}
%\usepackage{showframe}
\usepackage{lipsum}
\theoremstyle{cupplain}
\newtheorem{theorem}{Theorem}[section]
\newtheorem{lemma}[theorem]{Lemma}
\newtheorem{corollary}[theorem]{Corollary}
\newtheorem*{Lemmma}{Lemma}
\newtheorem{proposition}[theorem]{Proposition}
\newtheorem{conjecture}[theorem]{Conjecture}
\newtheorem{question}[theorem]{Question}
\newtheorem{claim}[theorem]{Claim}

\theoremstyle{cupremark}
\newtheorem{remark}[theorem]{Remark}
\newtheorem{other}[theorem]{Assumptions}
\newtheorem{assumption}[theorem]{Assumptions}
\newtheorem{example}[theorem]{Example}%[chapter]
\newtheorem{Construction}[theorem]{Construction}%[chapter]

\theoremstyle{cupdefinition}
\newtheorem{definition}[theorem]{Definition} %neu
\newtheorem{exercise}{Exercise}

\theoremstyle{cupproof}
\newtheorem*{proof}{Proof}

\numberwithin{equation}{section}

\articletype{RESEARCH ARTICLE}
\jrarticle{Gorge Waterway Action Society}
\jyear{2024}

\raggedbottom

\begin{document}

\begin{Frontmatter}

\title[Article Title]{POLLINATOR INSECTS IN ESQUIMALT GORGE PARK}

\author{A. Pinch}

%\author{E. Moellenhoff}
%\author{C. Ramsey}
%\author{C. Nelson}
%\author{H. Hickli}
%\author{S. Brescia}
%\author{N. Wood}
%\author{S. Gurney}

\authormark{A. Pinch}

\revised{September 6, 2022}

\keywords[Keywords]{ecology\sep restoration\sep insects\sep pollinators}

\abstract{Abstracts should be at most 250 words. It must be able to stand alone and so cannot contain citations to the paper's references, equations, etc. An abstract must consist of a single paragraph and be concise. Because of online formatting, abstracts must appear as plain as possible.}

\end{Frontmatter}

\section[This is an A Head]{This is an A head this is an A head this is an A head this is an A head  this~is~an~A~head this is an A head}\label{sec2}
\lipsum[1]

\begin{definition}[{See \cite{r5}}]\label{definition1}\lipsum[4]\end{definition}

\begin{remark}\lipsum[5] 
\begin{align*} V^{*I}(f^1_{\mathbf{w}},\ldots,f^{k_0}_{\mathbf{w}}):=\{\mathbf{z}\in\mathbb{C}^{*I}\mid f^{1,I}_{\mathbf{w}}(\mathbf{z}) = \cdots = f^{k_0,I}_{\mathbf{w}}(\mathbf{z})=0\}. \end{align*}
\lipsum[6]\end{remark}

\begin{assumption}\label{asm1}\lipsum[7]\end{assumption}

\lipsum[8]

\begin{lemma}\label{lemma1}\lipsum[9] 
\end{lemma}

\lipsum[10]

\begin{theorem}\label{theorem1}\lipsum[11]\end{theorem}

\lipsum[12]

\begin{corollary}\label{corollary1}\lipsum[13]\end{corollary}

\begin{proof}\lipsum[14]\end{proof}

\section[This is an A Head]{This is an A head this is an A head this is an A head this is an A head  this~is~an~A~head this is an A head}
\lipsum[15]

\subsection{This is a B head this is a B head this is a B head this is a B head this is a B head this~is~a~B head this~is~a~B head}
\lipsum[16]

\subsubsection{This is a C head this is a C head this is a C head this is a C head is a C head is a C~head~is~a~C head this is a C head}
\lipsum[2]\footnote{This is sample for footnote this is sample for footnote this is sample for footnote  this is sample for footnote this is sample for footnote.}

\section{Equations}

Equations in \LaTeX{} can either be inline or on-a-line by itself. For
inline equations use the \verb+$...$+ commands. Eg: The equation
$H\psi = E \psi$ is written via the command $H \psi = E \psi$.

For on-a-line by itself equations (with auto generated equation numbers)
one can use the equation or eqnarray environments \textit{D}.
\begin{equation}
\mathcal{L} = i {\psi} \gamma^\mu D_\mu \psi
    - \frac{1}{4} F_{\mu\nu}^a F^{a\mu\nu} - m {\psi} \psi
\label{eq1}
\end{equation}
where,
\begin{align}
D_\mu &=  \partial_\mu - ig \frac{\lambda^a}{2} A^a_\mu
\nonumber \\
F^a_{\mu\nu} &= \partial_\mu A^a_\nu - \partial_\nu A^a_\mu
    + g f^{abc} A^b_\mu A^a_\nu
\label{eq2}
\end{align}
Notice the use of \verb+\nonumber+ in the align environment at the end
of each line, except the last, so as not to produce equation numbers on
lines where no equation numbers are required. The \verb+\label{}+ command
should only be used at the last line of an align environment where
\verb+\nonumber+ is not used.
\begin{equation}
Y_\infty = \left( \frac{m}{\textrm{GeV}} \right)^{-3}
    \left[ 1 + \frac{3 \ln(m/\textrm{GeV})}{15}
    + \frac{\ln(c_2/5)}{15} \right]
\end{equation}
The class file also supports the use of \verb+\mathbb{}+, \verb+\mathscr{}+ and
\verb+\mathcal{}+ commands. As such \verb+\mathbb{R}+, \verb+\mathscr{R}+
and \verb+\mathcal{R}+ produces $\mathbb{R}$, $\mathscr{R}$ and $\mathcal{R}$
respectively.
\section{Figures}

As per the \LaTeX\ standards eps images in \verb!latex! and pdf/jpg/png images in
\verb!pdflatex! should be used. This is one of the major differences between \verb!latex!
and \verb!pdflatex!. The images should be single page documents. The command for inserting images
for latex and pdflatex can be generalized. The package that should be used
is the graphicx package.

\begin{figure}[t]
\FIG{\includegraphics{Fig1.eps}}{\caption{This is a widefig. This is an example of long caption this is an example of long caption  this~is~an example of long caption this is an example of long caption}
\label{fig1}}
\end{figure}


\section{Tables}

Tables can be inserted via the normal table and tabular environment. To put
footnotes inside tables one has to use the additional ``fntable" environment
enclosing the tabular environment. The footnote appears just below the table
itself.


\begin{table}[t]
\tabcolsep=0pt%
\TBL{\caption{Tables which are too long to fit,
should be written using the ``table*'' environment as shown here\label{tab2}}}
{\begin{tabular*}{\textwidth}{@{\extracolsep{\fill}}lccccc@{}}\toprule
\multicolumn{1}{@{}l}{\TCH{Level:}} & \multicolumn{1}{c}{\TCH{1}} & \multicolumn{1}{c}{\TCH{2}} & \multicolumn{1}{c}{\TCH{3}} & \multicolumn{1}{c}{\TCH{4}} & \multicolumn{1}{c}{\TCH{5}} \\\midrule
\multicolumn{1}{@{}l}{$g(\mathcal {T}_1(n))$} & \multicolumn{1}{c}{6} & \multicolumn{1}{c}{66} & \multicolumn{1}{c}{624} & \multicolumn{1}{c}{5,700} & \multicolumn{1}{c}{51,546} \\
\multicolumn{1}{@{}l}{$a(\mathcal {T}_1(n))$} & \multicolumn{1}{c}{4} & \multicolumn{1}{c}{25} & \multicolumn{1}{c}{214} & \multicolumn{1}{c}{1,915} & \multicolumn{1}{c}{17,224} \\
\multicolumn{1}{@{}l}{$a(\mathcal {T}_2(n))$} & \multicolumn{1}{c}{3} & \multicolumn{1}{c}{24} & \multicolumn{1}{c}{213} & \multicolumn{1}{c}{1,914} & \multicolumn{1}{c}{17,223} \\
\multicolumn{1}{@{}l}{$a(\mathcal {T}_3(n))$} & \multicolumn{1}{c}{3} & \multicolumn{1}{c}{24} & \multicolumn{1}{c}{213} & \multicolumn{1}{c}{1,914} & \multicolumn{1}{c}{17,223} \\
\multicolumn{1}{@{}l}{$\delta _7(\mathcal {T}_1(n))$} & \multicolumn{1}{c}{4} & \multicolumn{1}{c}{4} & \multicolumn{1}{c}{4} & \multicolumn{1}{c}{4} & \multicolumn{1}{c}{4} \\
\multicolumn{1}{@{}l}{$\delta _7(\mathcal {T}_2(n))$} & \multicolumn{1}{c}{3} & \multicolumn{1}{c}{3} & \multicolumn{1}{c}{3} & \multicolumn{1}{c}{3} & \multicolumn{1}{c}{3} \\
\multicolumn{1}{@{}l}{$\delta _7(\mathcal {T}_3(n))$} & \multicolumn{1}{c}{3} & \multicolumn{1}{c}{3} & \multicolumn{1}{c}{3} & \multicolumn{1}{c}{3} & \multicolumn{1}{c}{3} \\\botrule
\end{tabular*}{}}
\end{table}

\section{Cross referencing}

Environments such as figure, table, equation, align can have a label
declared via the \verb+\label{#label}+ command. For figures and table
environments one should use the \verb+\label{}+ command inside or just
below the \verb+\caption{}+ command.  One can then use the
\verb+\ref{#label}+ command to cross-reference them. As an example, consider
the label declared for Figure \ref{fig1} which is
\verb+\label{fig1}+. To cross-reference it, use the command
\verb+ Figure \ref{fig1}+, for which it comes up as
``Figure \ref{fig1}".
The reference citations should used as per the "natbib" packages. Some sample citations:  \cite{r1,r2}.

\section{Lists}
List in \LaTeX{} can be of three types: enumerate, itemize and description.
In each environments, new entry is added via the \verb+\item+ command.
Enumerate creates numbered lists, itemize creates bulleted lists and
description creates description lists.
List in \LaTeX{} can be of three types: enumerate, itemize and description.
In each environments, new entry is added via the \verb+\item+ command.
Enumerate creates numbered lists, itemize creates bulleted lists and
description creates description lists.
\begin{enumerate}[1.]
\item This is the 1st item
\item Enumerate creates numbered lists, itemize creates bulleted lists and
description creates description lists.
\item Numbered lists continue.
\end{enumerate}
List in \LaTeX{} can be of three types: enumerate, itemize and description.
In each environments, new entry is added via the \verb+\item+ command.
\begin{itemize}
\item This is the 1st item
\item Itemize creates bulleted lists and
description creates description lists.
\item Bullet lists continue.
\end{itemize}

\begin{Backmatter}
\paragraph{Acknowledgment}
Donec et nisl at wisi luctus bibendum. Nam interdum tellus ac libero.
\paragraph{Data Availability Statement}
Nulla non mauris vitae wisi posuere convallis. Sed eu nulla nec eros scelerisque pharetra.

\def\refname{References}
\begin{thebibliography}{99}
\expandafter\ifx\csname natexlab\endcsname\relax\def\natexlab#1{#1}\fi
\def\au#1{#1} \def\ed#1{#1} \def\yr#1{#1}\def\at#1{#1}\def\jt#1{{#1}}
\def\bt#1{#1}\def\bvol#1{\textbf{#1}} \def\vol#1{#1} \def\pg#1{#1}
\def\publ#1{#1}\def\arxiv#1{#1}\def\org#1{#1}\def\st#1{\textit{#1}}
\item[]\hspace*{-1.5pc}{This is a sample reference list, please use this to style your references.}
\bibitem{r1}
\hypertarget{r1}{}
{\au{D. N. {Bernstein}}},  \at{\textit{The number of roots of a system of equations}},  \jt{Funktsional. Anal. i Prilozhen.} \bvol{9} (\yr{1975}),  \pg{{1}--{4}}. English translation: Functional Anal. Appl. \textbf{9} (1975), 183--185 (1976).


\bibitem{r2}
\hypertarget{r2}{}
{\au{C. {Eyral}} and \au{M. {Oka}}}, \textit{Non-degenerate locally tame complete intersection varieties and geometry of non-isolated hypersurface singularities}, J. Algebraic Geom. \bvol{31} (\yr{2022}),  \pg{{561}--{591}}.


\bibitem{r3}
\hypertarget{r3}{}
{\au{H. A. {Hamm}}},  \at{\textit{Lokale topologische Eigenschaften komplexer R\"{a}ume}},  \jt{Math. Ann.} \bvol{191} (\yr{1971}),  \pg{{235}--{252}}.


\bibitem{r4}
\hypertarget{r4}{}
{\au{H. A. {Hamm}} and \au{D. T. {L\^{e}}}},  \at{\textit{Un th\'{e}or\`{e}me de Zariski du type de Lefschetz}},  \jt{Ann. Sci. \'{E}c. Norm. Sup\'{e}r. (4)} \bvol{6} (\yr{1973}),  \pg{{317}--{355}}.


\bibitem{r5}
\hypertarget{r5}{}
{\au{A. G. {Kouchnirenko}}},  \at{\textit{Poly\`{e}dres de Newton et nombres de Milnor}},  \jt{Invent. Math.} \bvol{32} (\yr{1976}),  \pg{{1}--{31}}.


\bibitem{r6}
\hypertarget{r6}{}
{\au{J. {Milnor}}},  \jt{\textit{Singular Points of Complex Hypersurfaces}}, Ann. of Math. Stud. \textbf{61},  \publ{Princeton Univ. Press}, 
 \publ{Princeton, NJ}; Univ. Tokyo Press, Tokyo, \yr{1968}.


\bibitem{r7}
\hypertarget{r7}{}
{\au{M. {Oka}}},  \at{\textit{On the bifurcation of the multiplicity and topology of the Newton boundary}},  \jt{J. Math. Soc. Japan} \bvol{31} (\yr{1979}),  \pg{{435}--{450}}.


\bibitem{r8}
\hypertarget{r8}{}
{\au{M. {Oka}}},  \at{\textit{On the topology of the Newton boundary II (generic weighted homogeneous singularity)}},\break  \jt{J. Math. Soc. Japan} \bvol{32} (\yr{1980}),  \pg{{65}--{92}}.


\bibitem{r9}
\hypertarget{r9}{}
{\au{M. {Oka}}},  \at{\textit{On the topology of the Newton boundary III}},  \jt{J. Math. Soc. Japan} \bvol{34} (\yr{1982}),  \pg{{541}--{549}}.


\bibitem{r10}
\hypertarget{r10}{}
{\au{M. {Oka}}},  \jt{\textit{Non-Degenerate Complete Intersection Singularity}}, Actualit\'{e}s Math.,  \publ{Hermann}, 
 \publ{Paris}, \yr{1997}.


\bibitem{r11}
\hypertarget{r11}{}
{\au{R. {Remmert}}},  \at{\textit{Holomorphe und meromorphe Abbildungen komplexer R\"{a}ume}},  \jt{Math. Ann.} \bvol{133} (\yr{1957}),  \pg{{328}--{370}}.


\bibitem{r12}
\hypertarget{r12}{}
{\au{H. {Whitney}}},  \at{\textit{Tangents to an analytic variety}}, Ann. of Math. (2) \bvol{81} (\yr{1965}),  \pg{{496}--{549}}.
\end{thebibliography}
\end{Backmatter}


\affauthor{A. Pinch}
\affiliation{Department/Institute\\ University\\ Address\\ City\\ Country\protect\email{email:}}

\affauthor{E. Moellenhoff}
\affiliation{Department/Institute\\ University\\ Address\\ City\\ Country\protect\email{email:}}

\end{document}
